\section{Abstract}
\textsc{Background:} Most maternal deaths are avoidable.delays in recognizing the need to seek care, delays in accessing  health care facilities, and delays in receiving adequate care can all make delivery of effective maternal health care practices very difficult. In recent years, mobile phones have grown in popularity for improving disease prevention and management, especially in the field of maternal and child health.A new system called Baby Monitor has attempted to address the delays in maternal health care delivery by taking pre- and post-natal screenings directly to mothers by using voice and text interactions over the phone. \\
\textsc{Objective:} The intent of this study was to design and pilot a mobile-phone based patient management system that served the needs and challenges of its end-users: community health volunteers (CHVs) and clinic nurses.. \\
\textsc{Methods:} This project combined elements of these previously established works by engaging health care providers and adapting MAMA messages to build upon the already existing, patient-centered Baby Monitor platform. Using a human-centered design framework, community health volunteers and clinic nurses helped shape the system and evaluated the pilot system for usability.\\
\textsc{Results:} The patient management system was found to be highly usable, with 94\% of respondents agreeing with the notion that the system helped them do their jobs better. \\\
\textsc{Keywords:} maternal health, infant health, mHealth, patient referral, health informatics