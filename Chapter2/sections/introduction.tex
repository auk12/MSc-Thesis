\section{Introduction}
\paragraph{Each year, more than 300,000 of women die from complications related to pregnancy, childbirth, or abortion. According to the WHO, most maternal deaths occur between the third trimester and the first six weeks after delivery\todo{add that most deaths occur during delivery or within first few days}, with the most common causes being severe bleeding, hypertensive diseases, and infections \citep{WHO2012}. Moreover, the burden of maternal mortality is greatest among developing countries where most poor women deliver at home. In sub-Saharan Africa, one in every sixteen women will die of pregnancy-related causes - a lifetime risk higher than anywhere else in the world \citep{Ronsmans2006}.}

\paragraph{Most maternal deaths are avoidable. At least eighty percent of maternal deaths can be prevented by a set of proven interventions provided by a skilled practitioner, while two-thirds of all infant deaths can be prevented with antenatal care provided by a health professional during the first six weeks after delivery. However, delays in recognizing the need to seek care, delays in accessing  health care facilities, and delays in receiving adequate care can all make delivery of the aforementioned interventions extremely challenging \citep{Thaddeus1994}.}

\todo{i've never used paragraph tags in this manner. maybe rstudio adds them for me when i compile from rnw file}\paragraph{These delays disproportionately affect women and families living in rural or remote regions. Community health workers or other health care professionals are few and far between in these areas, and complications that go unnoticed or are not treated early can prove to be deadly\todo{add stat on percentage of compilcations and note that they are often hard to predict}. The traditional solution to this challenge has been to increase the number of lay personnel, but there are many barriers to training and retaining human resources.\todo{cite intrapartum strategy and debate about how to invest: more emoc or skilled attendants at birth. need to better set up the importance of referral system in both.} A new automated screening and referral system called Baby Monitor is attempting to overcome this barrier by taking clinical screening directly to women using mobile phones. Women listen to screening questions in their local language and respond by pressing numbers on their keypads.\todo{i'd remove from ``by'' to end and introduce details later. specific details introduced here do not support referral piece, so it's not the best way to end this paragraph.} }

\paragraph{Over the past decade, mobile phones have had an incredible impact on low to middle income countries. Mobile phone technology has allowed millions of people to communicate to and from some of the most poor and remote areas of the world - especially in sub-Saharan Africa \citep{Adler2007}. \todo{insert mobile money and cite mpesa specfically before moving into health. that's where most progress has been made, certainly most uptake.}In recent years, as mobile phone penetration has continued to increase, the use of mobile technologies for health monitoring and management has also become increasingly popular. Specifically, studies have shown that mobile applications may be the most promising\todo{not sure about ``most promising'' because we don't know compared to what. maybe just ``are a promising''.} way to improve disease prevention and management, especially in developing countries\citep{ColeLewis2010}.}

\paragraph{Text messaging, due to its availability, low cost, and instantaneous nature, has been by far the most popular intervention used in mobile health programs. Previous literature has focused on text message reminders and their utility for improving health seeking behaviors \citep{ColeLewis2010}, clinical attendance \citep{Guy2012},  adherence to antiretroviral regimens for patients with HIV \citep{Horvath2012}, and self- management of diabetes care\citep{Krishna2008}. Although data remains relatively scarce, meta-analyses on each of the previously described areas have shown that text messaging interventions can have a positive impact on health behaviors and outcomes.\todo{cite}}

\paragraph{Mobile health initiatives have also focused on maternal and child health albeit in a limited context\todo{need to work in lavanya's paper}. Most of the current literature on mobile health for maternal and child health has focused on using mobile health interventions\todo{not sure what you are saying with ``focused on mobile health interventions''}, such as text messaging, to educate intermediate health care providers. A 2012 systematic review of 34 different studies on mobile health interventions for maternal child health revealed that the majority of research initiatives have targeted community health workers, skilled birth attendants, and midwives \citep{Tamrat2012}. Other studies have explored how text messaging can be used to educate midwives, birth attendants, or community health workers in rural areas \citep{Woods2012}.}\todo{i think we need a better structure here. maybe funnel from technology, e.g., sms, to end-user, e.g., chw}

\paragraph{Initiatives that have focused on mothers as end-users have also used text messaging as a means for education. The Mobile Alliance for Maternal Action (MAMA), a partnership between USAID and Johnson \& Johnson, has used text messages as the main tool to provide women with health information \citep{McCartney2012}. MAMA is a free text messaging service \todo{check this. i thought they just provided the content, not the service}that provides educational information to women during pregnancy and one year post-delivery. This program has been implemented in several developing countries, including India, South Africa, and Bangladesh, and has been customized for each target region based on the known cultural norms and beliefs regarding pregnancy and child care \citep{McCartney2012}. These programs may also help improve the overall patient experience for pregnant women who have opted to receive prenatal care. Studies have shown that pregnant women who received biweekly text messages offering support during the time between prenatal care visits had higher satisfaction levels with their care than women who did not receive any messages in between visits \citep{Jareethum2008}.} 

\paragraph{This project combined elements of these previously established works by engaging health care providers and adapting MAMA messages to build upon the already existing, patient-centered Baby Monitor platform. The intent of this study was to design and pilot a mobile-phone based patient management system that served the needs and challenges of its end-users: community health volunteers (CHVs) and clinic nurses.}\todo{i think you can expand this paragraph to synthesize the gaps, introduce baby monitor (all but technical details like verboice), and then articulate the motivation for this focus on referrals.}