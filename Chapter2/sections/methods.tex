\section{Methods}

\paragraph{The development process for the patient management component of Baby Monitor was driven by the philosophy\todo{hmmm, philosophy sounds not scientific maybe philosopy and methods?} of human-centered design. Within this framework, a product is iteratively designed specifically with the end-users' \todo{add needs}behaviors and preferences in mind, so as to create a system that is easy to learn and intuitive to use \citep{Oviatt2006}. In this case, CHVs were identified as the primary end users for a potential patient management system given their critical roles within the Kenyan health system.}\todo{i'd make this last sentence a new paragraph and describe the kenyan system in a few sentences. alternatively, and maybe preferably, add this to the introduction. if you do the latter, this sentence will have context.}

\paragraph{The first phase of the design process sought to understand how people and information flow within the currently existing health infrastructure.\todo{could use a ``for instance'' here} This phase also aimed to identify areas of need or difficulty for CHVs and nurses in completing their jobs that could be addressed by a potential patient management system. The second phase of the design process was focused on development of a mobile phone-based system that would address the challenges and needs identified in phase one and improve communication between patients, CHVs, and nurses so as to improve overall health outcomes. The third and final phase of the process focused on the evaluation of the system by the stakeholders themselves through a mobile phone-based usability survey.}\todo{placeholder here for possible addition of use metrics}

\subsection{Setting}

\paragraph{The study was centered at Sinoko Dispensary, a rural Level 2 health facility in the Ndivisi Division of Bungoma East District in Western Province, Kenya\todo{i think we need to reference the new units that came out of the new constitution. provinces have been dissolved. counties are the new first level. we are in bungoma county. ndivisi is still the division.}. Located approximately 2km off of the nearest paved road, Sinoko Dispensary is one of only three public health facilities in the area \todo{define} equipped to handle deliveries. The two remaining facilities - Webuye District Hospital and Webuye Health Center - are located within the nearby town of Webuye, located at the southwestern border of the Division.}\todo{include distance}

\subsection{Recruitment}
\paragraph{For nurses and CHVs to participate in the study, they were required to be comfortable speaking in both English and Swahili and comfortable using a mobile phone to receive calls and text messages.}

\paragraph{At the time of recruitment, the staff at Sinoko included one clinical officer, who served as the head administrator, and four nurses.\todo{maybe a footnote to explain positions. not all nurses had same level.} 55 CHVs also reported to Sinoko at least once per month to provide information on the families living in their villages within the Sinoko catchment area.\todo{define and talk about community units as part of community strategy} Of these providers, three nurses and six CHVs, each representing a different village, were selected to participate based on the inclusion criteria and interest in the project. Upon selection, verbal and written informed consent was obtained from the nurses and CHVs prior to study participation.}

\subsection{Phase One - Relevance}\todo{i wonder if we should use the same HCD headings: hear, create, deliver...always good to anchor in terms of methodology}
\paragraph{In order to better understand the role of CHVs local to Sinoko, two focus group discussions were conducted at the clinic with the six CHVs selected to participate in the study. In the first discussion, the CHVs were asked to describe their daily workflow, discuss their experiences working with pregnant women and new mothers, and detail their administrative responsibilities. They were also asked to identify the most challenging aspects of their jobs as CHVs and to describe some of the local attitudes and perceptions related to pregnancy and maternal and child health. The second discussion was more focused on the concept of patient referral. Participants were asked to collectively describe their ideal system of communication between patients, CHVs, and nurses at the clinic. Audio from these discussions was recorded and analyzed for potential themes for design features for the patient management system. }

\paragraph{After the focus group discussion, field visits \todo{i think this needs to be more active to represent what you did. really shadowing, right?}were scheduled with two of the participating CHVs on separate dates. The purpose of these visits was to gain a better understanding of the CHVs daily responsibilities and to identify potential ways for the patient management system to fit into their existing workflow. Number of patients seen per day, amount of time spent with each patient, primary concern or chief complaint, and patient referral status (i.e. whether the patient was referred to Sinoko or scheduled for a follow-up home visit from the CHV) were documented for each patient visited over the course of the day.}

\paragraph{The final element of this design phase was a focus group discussion with the Sinoko clinic nurses selected to participate in the study. They were asked to describe their work responsibilities at the clinic, their experiences working with pregnant women and new mothers, and their interactions with the local CHVs. Like the CHVs, the nurses were also asked to  describe their ideal system of communication between patients, CHVs, and the clinic. This discussion was also recorded and analyzed to identify themes and design principles.}


\subsection{Phase Two - Development}\todo{if you introduce IVR at the end of the intro, then can assume reader remembers here. note that i am adding text directly to the document.}
\paragraph{With an understanding of user needs, behaviors, and preferences, we began the process of developing the referral component of Baby Monitor. The Baby Monitor service integrates several technologies: Verboice, a platform for designing and initiating automated phone calls over the internet; a Voice Over Internet Protocol (VoIP) provider in Kenya; a software framework called Asterisk used to connect Verboice to the VoIP provider; a telecommunications company in Kenya that delivers the automated call to the mobile handset of the end-user; a local SMS\todo{be sure to define earlier} gateway provider that sends text messages to end-users; and an analysis engine to process call data and trigger new calls from Verboice and send text messages from the SMS gateway provider. SOMETHING ABOUT INTEROPERABILITY.} 

\todo{i'd recommend mini-headings for each component}\paragraph{The system was designed in Verboice, an open source platform for creating projects that interact with end-users via voice and text, and R, an open source statistical computing environment. Verboice allows end-users to listen to audio messages in multiple languages, respond to questions with the phone keypad, and  record their own voice messages. Using the web-based Verboice platform, the research team built upon the existing Baby Monitor platform to create call flows designed for use by CHVs at Sinoko. Each call flow consisted of a series of instructions, questions, and prompts that require numeric input from the user's phone keypad, and was designed to address the design principles and themes identified for the patient management system during the first phase of the design process. For questions that required a 'yes' or 'no' answer, users were asked to press '1' or '3' on their keypads. For other questions, users were also asked to enter numerical data through their keypads. No data or answers to questions were stored locally on their phones; all responses to all questions were saved to the research team's Verboice database.} 

\paragraph{The research team also created a set of text messages specific to the roles and responsibilities of the CHVs in order to supplement the interactive voice response system. These messages were designed to use information provided by the CHVs in previous calls with the system to help them complete their daily responsibilities. Additionally, the research team adapted a set of text messages from the Mobile Alliance for Maternal Action (MAMA) designed for pregnant women and new mothers. Both sets of text messages were automated through an R script written for the larger Baby Monitor project, which also automated calls to the CHVs through Verboice.}

\paragraph{In order to test these call flows and automated text messages, the research team conducted a mock testing session with the CHV focus group. Index cards with text were used to represent each audio or text message, and volunteers were selected to read the messages aloud to the group. This was done in order to confirm the content and logical flow of the messages and questions, and to gain feedback on the strengths and weaknesses of the system. Based on feedback from this focus group session, the research team finalized the content and flow of each message in the call flow within the web-based Verboice platform. A woman native to Ndivisi and familiar with the local dialects was recruited to assist in translation of all messages and recording of the audio messages in English and Swahili. Recording was completed at a studio in a nearby town.}


\subsection{Phase Three - Evaluation}
\paragraph{The three nurses previously selected to participate in the study and the full sample of 55 CHVs were chosen to pilot the patient management system with patients within the Sinoko catchment area. The primary outcomes for this evaluation phase were frequency of use of the system and user-determined\todo{replace with self-reported} usability rating. Data regarding the use of the patient management system was collected over the course of six months, after which usability testing was initiated. A modified version of the Health IT Usability Evaluation Scale \citep{Yen2010} was administered to all CHVs through a Verboice call flow (see Appendix A). Participants were called through Verboice via an automated R script and listened to a series of statements regarding the quality of work life, perceived usefulness, and perceived ease of use of the system. Using their numeric keypads, they were asked to press '1' to agree with the statement and '3' to disagree. They were subsequently asked to whether they agreed or disagreed 'a lot' or 'a little'. This modified Likert scale allowed for a quantification of the system's overall usability and identification of weaknesses in the current system design.\todo{since we could have asked them to use the keys 1-4, design was not a limitation. ease of understaning and administration was the reason.}}


