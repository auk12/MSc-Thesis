\abbreviations

% You can put here what you like, but here's an example
%Note the use of starred section commands here to produce proper division
%headers without bad '0.1' numbers or entries into the Table of Contents.
%Using the {\verb \begin{symbollist} } environment ensures that entries are
%properly spaced.

%\section*{Symbols}
%
%Put general notes about symbol usage in text here.  Notice this text is
%double-spaced, as required.
%
%\begin{symbollist}
%	\item[$\mathbb{X}$] A blackboard bold $X$.  Neat.
%	% Optional item argument makes the symbol/abbr
%	\item[$\mathcal{X}$] A caligraphic $X$.  Neat.
%	\item[$\mathfrak{X}$] A fraktur $X$.  Neat.
%	\item[$\mathbf{X}$] A boldface $X$.
%	\item[$\mathsf{X}$] A sans-serif $X$. Bad notation.
%	\item[$\mathrm{X}$] A roman $X$.
%\end{symbollist}
%
%\section*{Abbreviations}
%
%Long lines in the \texttt{symbollist} environment are single spaced, like in
%the other front matter tables.

\begin{symbollist}
	\item[CHV] Community health volunteer
	\item[CHW] Community health worker
	\item[CHEW] Community health extension worker
	\item[DHS] Demographic and Health Survey
	\item[IVR] Interactive voice response
	\item[MAMA] Mobile Alliance for Maternal Action
	\item[mHealth] Mobile health
	\item[MOTECH] The Mobile Technology for Community Health program in Ghana
	\item[m-Pesa] Mobile money service utilized in Kenya
	\item[SMS] Short message service, used interchangeably with text messaging service
	\item[SPA] Service Provision Assessment
	\item[VoIP] Voice over Internet Protocol
	\item[WHO] World Health Organization
\end{symbollist}
