\chapter{Overview}

\paragraph{Each year, more than 300,000 of women die from complications related to pregnancy, childbirth, or abortion. According to the WHO, most maternal deaths occur between the third trimester and the first six weeks after delivery, with the most common causes being severe bleeding, hypertensive diseases, and infections \citep{WHO2012}. Moreover, the burden of maternal mortality is greatest among developing countries where most poor women deliver at home. In sub-Saharan Africa, one in every sixteen women will die of pregnancy-related causes - a lifetime risk higher than anywhere else in the world \citep{Ronsmans2006}.}

\paragraph{Most maternal deaths are avoidable. At least eighty percent of maternal deaths can be prevented by a set of proven interventions provided by a skilled practitioner, while two-thirds of all infant deaths can be prevented with antenatal care provided by a health professional during the first six weeks after delivery. However, delays in recognizing the need to seek care, delays in accessing  health care facilities, and delays in receiving adequate care can all make delivery of the aforementioned interventions extremely challenging \citep{Thaddeus1994}.}

\paragraph{These delays disproportionately affect women and families living in rural or remote regions. Community health workers or other health care professionals are few and far between in these areas, and complications that go unnoticed or are not treated early can prove to be deadly. The traditional solution to this challenge has been to increase the number of lay personnel, but there are many barriers to training and retaining human resources. A new automated screening and referral system called Baby Monitor is attempting to overcome this barrier by taking clinical screening directly to women using mobile phones. Women listen to screening questions in their local language and respond by pressing a key. Baby Monitor assesses responses and, when fully operational, will send information, make referrals, and dispatch community health workers.}

\paragraph{Over the past decade, mobile phones have had an incredible impact on low to middle income countries. Mobile phone technology has allowed millions of people to communicate to and from some of the most poor and remote areas of the world - especially in sub-Saharan Africa \citep{Adler2007}. In recent years, as mobile phone penetration has continued to increase, the use of mobile technologies for health monitoring and management has also become increasingly popular. Specifically, studies have shown that mobile applications may be the most promising way to improve disease prevention and management, especially in developing countries\citep{ColeLewis2010}.}

\paragraph{Text messaging, due to its availability, low cost, and instantaneous nature, has been by far the most popular intervention used in mobile health programs. Previous literature has focused on text message reminders and their utility for improving health seeking behaviors \citep{ColeLewis2010}, clinical attendance \citep{Guy2012},  adherence to antiretroviral regimens for patients with HIV \citep{Horvath2012}, and self- management of diabetes care\citep{Krishna2008}. Although data remains relatively scarce, meta-analyses on each of the previously described areas have shown that text messaging interventions can have a positive impact on health behaviors and outcomes.}

\paragraph{Mobile health initiatives have also focused on maternal and child health albeit in a limited context. Most of the current literature on mobile health for maternal and child health has focused on using mobile health interventions, such as text messaging, to educate intermediate health care providers. A 2012 systematic review of 34 different studies on mobile health interventions for maternal child health revealed that the majority of research initiatives have targeted community health workers, skilled birth attendants, and midwives \citep{Tamrat2012}. Other studies have explored how text messaging can be used to educate midwives, birth attendants, or community health workers in rural areas \citep{Woods2012}.}

\paragraph{The few initiatives that have focused on mothers as end-users, instead of health care providers, have also used text messaging as a means for education. The Mobile Alliance for Maternal Action (MAMA), a partnership between USAID and Johnson \& Johnson, has used text messages as the main tool to provide women with health information \citep{McCartney2012}. MAMA is a free text messaging service that provides educational information to women during pregnancy and one year post-delivery. This program has been implemented in several developing countries, including India, South Africa, and Bangladesh, and has been customized for each target region based on the known cultural norms and beliefs regarding pregnancy and child care \citep{McCartney2012}. These programs may also help improve the overall patient experience for pregnant women who have opted to receive prenatal care. Studies have shown that pregnant women who received biweekly text messages offering support during the time between prenatal care visits had higher satisfaction levels with their care than women who did not receive any messages in between visits \citep{Jareethum2008}.} 


\section{Fieldwork Site}
Eldoret, Kenya is very beautiful.

\section{Research Methods}
Methods were rigorous.