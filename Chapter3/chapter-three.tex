\chapter{Chapter Three}

%This chapter will  summarize the overall project, with a focus on lessons learned, implications for future research and intervention, and limitations. It can, but does not have to, include a more personal reflection on the research process. 

\section{Lessons Learned}


\begin{description}
	\item[Voice recordings] \hfill \\
	Description of lessons learned related to recording voice messages. Voice clarity is critically important for interactive voice response technologies, as it is the primary method of engaging with the end user. Also enables access to a userbase with low literacy. So clarity of voice messages is critical for success of IVR. Background noise, poor quality made recording very difficult at first. 
	\item[Translation of messages] \hfill \\
	The first description of translation problems.
	\item[Simplicity of call flows and messages] \hfill \\
	Keeping voice and text messages short and simple was crucial. Allowed interactions to be quick and easy to understand. 
	\item[Mobile network variability] \hfill \\
	Description of mobile network variability. 
	
\end{description}


\section{Limitations}
This study piloted an intervention that was previously untested in this target population. Thus, the scope of the study was limited to a single clinic and a small, convenience sample of CHVs working in the clinic's catchment area. Participants for focus groups were selected based on English comprehension and demonstrated interest in the study, and thus may not generally represent the perspectives or viewpoints of all other CHVs within the health system. Due to time contraints, only one cycle of the Hear and Create phases were completed; in future iterations, additional cycles of prototyping and mock testing would have been conducted in order to refine and create additional features for the system.  

\section{Implications for Future Research}
Future studies can build upon these findings by expanding the scope of the study beyond one study clinic and its corresponding two community units. Additional focus groups from the CHV and nurse population would also contribute to a more robust exploration of the flow of information and people with respect to maternal and child healthcare at the community level in this region. From a design perspective, upcoming versions of the system will also aim to further integrate the Baby Monitor screening service with the patient management component, so as to alert CHVs and suggest steps for follow up when a woman presents with concerning screening results. This will enable CHVs to identify high-risk patients and to adjust their approach in visiting and caring for them at the community level. Future iterations of the intervention will also evaluate additional features that were suggested by CHVs during the mock testing sessions in the Create phase, such as text message reminders for CHVs about upcoming home visits and expected delivery dates of women enrolled with the Baby Monitor system from their village. 


\section{Conclusions}
